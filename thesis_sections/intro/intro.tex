\documentclass[11pt]{article}
\title{\textbf{Quorum-sensing influence on extracellular lipase production in activated sludge}}
\author{Anna Liza Kretzschmar\\
        z3218219\\
        Supervisor: Mike Manefield}
\date{}

\usepackage[square,sort,comma,numbers]{natbib}
\usepackage{rotating}
\usepackage{multirow}

\begin{document}

\maketitle

\section{Introduction}
Wastewater treatment plants are common in the global urban setting and a necessity for the continued function of our society. Fluid waste enters the plant, is purified and the effluent can be re-introduced into the environment. It is necessary to understand the processes in order to optimise the functionality of the plant, as the global population expands and strains the limited water resources available. In 1914 Arden and Lockett pioneered the development of activated sludge, which is commonly employed today \cite{ardern1914experiments}.


The microbes in activated sludge, which are incubated with filtered waste influent, oxidise the sewage particulates \cite{Price_95}. Microbial consortia in activated sludge form a detached biofilm called a floc, however what seeds floc formation is still unknown. Inter bacterial communications induce transcriptomic change via a system called quorum-sensing, which plays a major role in a consortium's interaction\cite{parsek2005sociomicrobiology}. Hence our understanding of the phenotypes impacted by this communication is essential for optimising the process of water remediation \cite{singh2006biofilms}

\subsection{Wastewater treatment is essential to civilisation}
The development of the current municipal wastewater treatment system has evolved over millennia to combat the waste arising from congregated human habitation. The Mesopotamian empire, from between 3500 to 2500 BC, built connections from some homes to storm water drains to remove wastes, while the Babylonians developed clay pipes to fulfil the same purpose \cite{lofrano2010}. The logical necessity to develop such a system to prevent disease and minimise environmental impact globally and has been continuously improved. In 1914 Arden and Lockett pioneered the development of activated sludge, which is one of the most commonly employed today \cite{jenkins2004manual,muchie2010bioremediation}


Current plants combines physical, chemical and biological treatment strategies to the polluted influent in order to reduce environmental and health impacts. Hence these treatments are in place to minimise the presence of N, P and C; reduce the biological oxygen demand and chemical oxygen demand; as well as to filter out harmful substances and pathogens \cite{mayhew1997low}.
Water quality is in part assessed by the dissolved oxygen concentration, which reducing agents can unbalance. The parameters for this assessment are measured by the concentration of substances present that can be oxidised chemically or biologically to inorganic end products \cite{pisarevsky2005chemical}.

\subsubsection{\emph{Microbial floc formation underlies successful wastewater treatment}}
Initially, large solids in the influent are either filtered out or broken down into smaller particulates to prevent blockages or excessive mechanical strain during the process. During the primary treatment phase, the influent is deposited in a large settlement tank where the majority of the particles settle to the bottom due to gravitational force. The weir controlled outlet flow is directed to the aeration tank for secondary treatment. The activated sludge in the aeration tank comprises the biological aspect of the treatment. It is an oxygenated system where the influent is incubated with microbial consortia, predominantly affiliated with flocs, which live by oxidising the sewage particles in the mixed liquor \cite{mayhew1997low}. The suspension is drawn off into a final settlement tank where the sludge flocs settle at the bottom, and are pumped out and recycled back into the mixed liquor; or discarded. Depending on the required effluent quality, tertiary treatment may also be required to remove residual solids, compounds or planktonic bacteria that could pollute or enrich the effluent destination \cite{Price_95}.
% knowledgegap
% jesus fucking christ, more references

\subsubsection{\emph{Microbial composition determines wastewater treatment efficiency}}
To optimise the efficiency of wastewater treatment, understanding the composition of the microbial communities at work in activated sludge and how their interactions influence their capabilities of remediation is essential \cite{daims2006}.
Single celled organisms from all three domains of life reside in this system - protozoa, bacteria, nematodes, fungi and viruses. All of these play a role in degradation of solids and nutrient cycling and hence in wastewater remediation \cite{muchie2010bioremediation}. The domain most active in remediation are the bacteria \cite{spellman2008handbook}, which will be focused upon.
% mention foamng and sludge bulking

The advance of community analysis techniques, which exclude the necessity to culture microbes in the laboratory, has made it possible to investigate the composition of microbial communities, including in activated sludge. The most frequent target for operational taxonomic unit (OTU) analysis centres around the 16S ribonucleic acid (rRNA) gene, which is then compared to expansive 16S databases such as NCBI BLAST. Several techniques can be utilised around 16S rRNA, such as constructing a 16S rRNA gene library \cite{McGarvey_04}, analysing ribosomal intergenic spacer sequences \cite{Yu_01}, 16S-restriction fragment length polymorphism \cite{Gilbride_06} and comparing community structures via denaturing gradient gel electrophoresis (DGGE) \cite{Hesham_11} or pyrosequencing \cite{wang2012pyrosequencing}.


In wastewater treatment plants, the microbial community composition is strongly dependent on what type of waste influent the microbes are exposed to, as well as the operational settings, such as chemical oxygen demand \cite{Gilbride_06,wang2012pyrosequencing,hu2012microbial}. Municipal wastewater treatment plant influent varies in composition \citep{henze2002wastewater}.


\begin{sidewaystable}[!htbp]
\begin{tabular}{ | l | l | p{4.5cm} | p{7cm} | l | }
\hline
Study & Method & Sample size & Core microbiome composition & Refernce\\
\hline
Hesham et al. & DGGE & One plant with two different operational modes over 6 months, China & 2x \emph{$\alpha$-} \& 2x \emph{$\beta$- proteobacteria}; 3x \emph{bacteroidetes}; 2x \emph{actinobacteria}; 2x \emph{firmicutes} & \cite{Hesham_11} \\
\hline
Wagner et al. & DGGE & 750 16S rRNA sequences from wastewater treatment plants and reactors & \emph{$\alpha$-, $\beta$-} \& \emph{$\gamma$- proteobacteria}; \emph{bacteroidetes};\emph{actinobacteria} & \cite{Wagner_02} \\
\hline
Wang et al. & Pyrosequencing & 14 plants \& pilot/benchtop operations in China & 21 - 53 \% Proteobacteria, where \emph{$\beta$-, $\alpha$-} \& \emph{$\gamma$-proteobacteria} were present 21 - 52 \%, 7 - 48 \% and 8 - 34 \% respectively & \cite{wang2012pyrosequencing} \\
\hline
Hu et al. & Pyrosequencing & 12 plants where 4 were A/A/O, China & In A/A/O proteobacteria dominated, with up to 2:1 ratio to bacteroidetes. In other types of plants the dominant phylum proteobacteria was occasionally replaced by bacteroidetes & \cite{hu2012microbial} \\
\hline
Ranasinghe et al. & Pyrosequencing & 12 plants over two years, Japan & Proteobacteria represented 38 \% of total assigned reads where 15 \% belonged to $\beta$-proteobacteria & \cite{ranasinghe2012revealing} \\
\hline
Xia et al. & Microarray & Three plants in China and two in USA & Proteobacteria wast the dominant phylum at 50 \% to 62 \%, where $\gamma$-proteobacteria represented  31 \% to 38 \% and $beta$-proteobacteria 30 \% to 35 \% & \cite{xia2010bacterial} \\
\hline
 & & & & \\
\hline
\end{tabular}
\caption{Studies characterising core microbiomes found in activated sludge}
\end{sidewaystable}


\emph{Table xx} summarises methods used to analyse activated sludge microbial communities across geographical and operational settings, showing a core microbiome. Pyrosequencing and microarrays studies give a snapshot of the microbial community, including minority OUTs \cite{ranasinghe2012revealing}. DGGE gives an insight into the change in abundance of dominant organisms active under certain conditions and over time. 

\subsubsection{\emph{Samples were aquired from activated sludge in St. Mary's wastewater treatment plant}}
St Mary's municipal waste water treatment plant is where sample collection for this project took place. It is located in St Mary's in western Sydney serving a population of app. 160000 in a catchment area of 84 km$^{2}$ and discharges into the Hawkesbury-Nepean River. On a daily basis the plant processes about 35 million litres of wastewater. Wastewater can enter two influent streams - stage 1 and 2 or stage 3. The streams are subjected to differing primary treatments, however secondary and tertiary treatment is the same. The biological treatment is part of the secondary treatment stage, as can be seen in figure xx. The biological process is conducted in three stages (A/A/O): 1. Anaerobic zone, where microbes take up carbon and release phosphates; 2. Anoxic zone, where carbon is consumed and nitrates are released as nitrogen gas; 3. Aerobic zone, where nitrification of ammonia occurs and organic matter increasing the oxygen demand is reduced \cite{stmarys}.
\begin{figure}
\includegraphics[scale=0.9]{SMary_process.png}
\caption{Wastewater processing as employed by St mary's treatment plant \cite{stmarys}}
\end{figure}

\subsection{AHL mediated gene expression facilitates bacterial communication and co-operation}
The principle of QS refers to bacterial communication and behavioural change in response to extracellular molecules. These molecules diffuse in and out of the cells and trigger a change in transcriptomic regulation depending on their concentration. This allows the consortia, whether multi-species or cross-kingdom \citep{williams2007quorum}, to engage in a concerted effort that resembles multi-cellularity \cite{kjelleberg2002}. While there are several QS systems, this report focuses on AHL dependent QS only. The bacteria that facilitate this style of QS are predominately from the phylum proteobacteria, which are diverse and well represented in the environment and especially so in municipal activated sludge \cite{Hesham_11,Wagner_02}.  


QS machinery consists of two major constituents - LuxR and LuxI, or their respective homologs. The latter is an AHL synthase, while the former is a receptor that binds AHLs at a threshold. The complex functions as a transcription factor for \textit{luxRI} or homologs and species specific functional genes. These genes typically play a role in bacterial cell adhesion to surfaces, biofilm formation and the expression of extracellular enzymes \cite{Flemming_10}.
Furthermore, these traits are subject to activation, inhibition and degradation by external factors as well as potential cross regulation between several QS systems within the organism \cite{juhas2005}.


\subsubsection{\emph{AHL responsive gene circuits are found in Proteobacteria}}
The proteobacteria is the most prevalent phylum containing AHL dependent QS circuits \cite{gelencser_12}. In 2008 Case et al. compiled a list of isolates containing \emph{luxRI} homologs from 512 completed genomes on the NCBI platform, where 13 \% of those contained both circuit components. These 13 \% represent 26 \% of proteobacterial genomes within that subset \cite{case_08}.
The type II secretion system is only found in this phylum, usually associated with delivering QS influenced virulence factors and enzymes, such as lipase, into extracellular space \cite{sandkvist2001}. 
% explain lux box
When assessing the frequency in which Case's found QS active proteobacteria \cite{case_08} in light of the frequency this phylum is found in activated sludge \cite{Wagner_02,Hesham_11}, it follows that this system is relevant in the microbial consortia comprising activated sludge flocs. This is reflected in the change of enzyme expression in sludge isolates in response to AHL stimulation as detected by Chong et al \cite{Chong_12}. 

\subsubsection{\emph{The role of AHLs in Bioflocculation}}
Types of biofilm which do not adhere to a surface but rather float, are flocs in activated sludge \cite{wingender1999}. Bioflocculation is essential for wastewater treatment, as the settlement of the flocs is key to drawing off remediated water. QS is commonly found in the biofilm setting, which is an alternative bacterial lifestyle to the solitary planktonic one \cite{webb2003}. Biofilms constitute an aggregated community of microbes at the liquid-solid interface surrounded by secreted extracellular polymeric substances (EPS), consisting of polysaccharides, nucleic acids, lipids and proteins \cite{wingender1999}. EPS represents 85-90 \% of biofilm dry weight \cite{Frolund_96}.


The composition of the biofilm is dependent on the participating species coupled with the environment, and are sites of rapid adaptation through continuous natural selection for survival \cite{boles2008,matz2005,palmer2001}.  
The EPS offers the advantages of defence against predation by other bacteria \cite{rao2005} and protozoic grazing \cite{matz2005}; increased resistance against some chemicals such as antibiotics and hydrogen peroxide \cite{burmolle_06}; a stable matrix for the cells to reside in \cite{Flemming_10}; and to immobilise extracellular enzymes offering proximity while protecting them from proteolysis and conferring resistance to higher temperatures \cite{wingender2002extracellular,Flemming_10,skillman1998}.

Extracellular lipase expressed by the \emph{$\gamma$}-proteobacterium \textit{Pseudomonas aeruginosa} associates with alginate within the EPS which anchors the enzyme with weak bonding forces \cite{mayer1999role,wicker1987}. Hence the lipase  liberates fatty acids close to the cell.
% not necessary?: It has been widely documented that high lipid content of waste influent negatively affects flocculation, can prevent it all together. *ref*

\subsection{Lipids in wastewater need to be degraded by lipase}
Municipal wastewater contains lipids, at a concentration between 40 and 100 mg/m$^{3}$ \cite{Forster_92} presenting 31 \% of the chemical oxygen demand of domestic wastewater \cite{Raunkjaer_94}. Oleic acid is the most common fatty acid found in native as well as used olive and sunflower oils, which are commonly used in domestic settings \cite{haba2000isolation}.  An excess in lipid content has been shown to inhibit flocculation and promote the growth of filamentous bacteria, which are linked to sludge bulking - an event that induces foaming and reduces effluent quality \cite{Forster_92}.


Due to the hydrophobic nature of lipids, they may be associated with other particulates in the activated sludge or form a separate phase. In the former case, membrane or EPS bound lipases are likely expressed for degradation, whereas lipids in a separate phase could be targeted with extracellular lipases. It is advantageous for consortia to express lipase to utilise this carbon source and and lipase is commonly expressed in activated sludge \cite{gessesse2003lipase}. Investigations as to the most effective strategy for addressing lipids in wastewater have been conducted extensively. Strategies including pretreatment with large concentrations of lipases, introducing pure or mixed lipid degrading cultures and exposure activated sludge \cite{Wakelin_97}. Activated sludge showed the highest grease removing activity \cite{Wakelin_97}. 
%need more refs - extensive doesnt equal singular ref.




\subsubsection{\emph{Lipase are extracellular degradation units}}
Triacylglycerol acylhydrolase (Enzyme Class 3.1.1.3) catalyses the reversible hydrolysis of triacylglycerols by targeting the ester bonds that attach the fatty acid side chain to the glycerol backbone. The model lipid Glyceryl trioleate (GT) used in this study consists of three oleic acid side chains. Genes encoding for lipase will be denoted as \emph{lip}.
The activity of lipase is highly chemo-, enantio- and regioselective. Common to this class of enzyme is an extruding loop which extends over the active site, also commonly called the lid. The aggregated, hydrophobic nature of the substrate causes an effect termed interfacial activation, which causes the lid to move to reveal the underlying active site to the substrate \cite{derewenda1992,van_Tilbeurgh1993}. These features contribute to lipases being highly efficient biocatalysts in organic chemistry and their extensive representation in the industrial setting. 
They are found within the textile, detergent, food processing, leather, pharmaceutical, pulp and paper industries \cite{hasan_06}; are essential for the production of fine chemicals such as flavours, cosmetics, agrochemicals and therapeutics \cite{jaeger2002}; and are under investigation for their potentially to produce biodiesel as a replace fossil fuels \cite{hasan_06,iso2001}. 


Microbial lipases are secreted into the extracellular space, where they can catalyse the liberation of fatty acids at the lipid-water interface, which are then absorbed and utilised as a C source. In Gram negative bacteria, including proteobacteria, the lipase zymogen passes two membranes separated by the periplasm, before secretion and folding into it's active conformation \cite{bos2007,michel2009}. There are two pathways by which this can occur: the type I or type II secretory pathways. 


\subsubsection{\emph{AHL mediated gene expression of lipase in proteobacteria is convoluted}}
The influence of AHL mediated gene expression on lipase production has been documented in several genera, discussed in this section with their respective classification and \emph{luxRI} homologs as can be seen in \emph{table xxx}.

\begin{table}
\begin{tabular}{ | p{2.5cm} | p{3cm} | p{1.5cm} | p{1.5cm} | p{2.5cm} | }
\hline
Species & Taxonomy (class, order) & \emph{luxR} homologue & \emph{luxI} homologue & Reference/ GenBank accession \# \\
\hline
\emph{Burkholderia cepacia} & \emph{$\beta$-proteobacteria, Burkholderiales} & \emph{cepR} & \emph{cepI} & \cite{lewenza1999} \\
\hline
\emph{Burkholderia kururiensis} & \emph{$\beta$-proteobacteria, Burkholderiales} & \emph{braR} & \emph{braI} & \cite{suarez2008} \\
\hline
\emph{Burkholderia unamae} & \emph{$\beta$-proteobacteria, Burkholderiales} & \emph{unaR} & \emph{unaI} & \cite{suarez2010} \\
\hline
\emph{Burkholderia xenovorans} & \emph{$\beta$-proteobacteria, Burkholderiales} & \emph{“luxR homologue”} & - & NC007951.1 \\
\hline
\emph{Burkholderia glumae} & \emph{$\beta$-proteobacteria, Burkholderiales} & \emph{tofR} & \emph{tofI} & AB757840.1 \\
\hline
\emph{Burkholderia vietnamiensis} & \emph{$\beta$-proteobacteria, Burkholderiales} & \emph{} & \emph{} & \cite{conway_02}, \cite{ulrich2004}
 \\
\hline
\emph{Pseudomonas aeruginosa} & \emph{$\gamma$-proteobacteria, Pseudomonadales} & \emph{} & \emph{} & \cite{juhas2005} \\
\hline
\emph{Pseudomonas fluorescens} & \emph{$\gamma$-proteobacteria, Pseudomonadales} & \emph{phzR} & \emph{phzI} & L48616 \\
\hline
\emph{Xenorhabdus nematophilus} & \emph{$\gamma$-proteobacteria, Enterobacteriales} & \emph{“luxR homologue”} & - & FN667742.1 \\
\hline
\emph{Serratia marcescens} & \emph{$\gamma$-proteobacteria, Enterobacteriales} & \emph{smaR} & \emph{smaI} & AJ5980 \\
\hline
\emph{Serratia protemaculans} & \emph{$\gamma$-proteobacteria, Enterobacteriales} & \emph{sprR} & \emph{sprI} & AY040209.1 \\
\hline
\end{tabular}
\caption{The taxonomic classification and \emph{luxRI} homologs for species shown to have AHL mediated influence on lipase expression.}
\end{table}

\emph{\underline{Burkholderia spp.}} 
\\In \emph{Burkholderia cepacia}, Lewenza et al. have demonstrated that the expression of \emph{lipA} is linked to CepR, the LuxR homolog \cite{lewenza1999}, which is in direct opposition to the results obtained by Huber et al. \cite{huber2001}. In Lewenza et al.'s study, lipase production in \emph{cepR} mutants was reduced by up to 45 \%. However the supplementation of CepR via a plasmid did not restore wild type lipase expression, which suggests that the link between \emph{cepR} and \emph{lipA} is other than transcriptomal activation. The lack of a \emph{lux} box within \emph{lipA}, to which the CepR:AHL transcription factor would bind, supports the postulation that \emph{lipA} is situated downstream of \emph{cepR} and under same operon control \cite{lewenza1999}. 



\emph{B. glumae}, an emerging rice pathogen, regulates it's lipase production with AHL mediated QS and lipase expression is innately linked with the pathogenic phenotype. TofR, the LuxR homolog, activates \emph{lipA} transcription \cite{devescovi_07}. 


A study by Suarez-Moreno et al. which compared the plant associated species \emph{B. kururiensis}, \emph{B. unamae} and \emph{B. xenovorans} found no discernible impact of disabling the respective LuxR homologs on lipase production, in agreement with Huber et al. \cite{huber2001,suarez2010}. For these species tested, which live in symbiosis with the plant host, it follows that lipase expression, which is generally considered a virulent attribute, should not be population density dependent as increasing cell density would imply successful symbiosis. 



In \emph{B. vietnamiensis}, the literature on how QS influence lipase secretion are contradictory. Conway et al. recorded no detectable QS influence on lipase production whether in the wild type or an AHL synthase (\emph{bviI}) deficient mutant \cite{conway_02}. On the other hand Ulrich et al. conducted a they found three separate AHL synthases referred to as \emph{btaI123} and five separate \emph{luxR}-like transcriptional regulators named \emph{btaR12345}, which is a nomenclature system distinct to other nomenclature reported here. They found that \emph{btaR1}, \emph{btaR3}, \emph{btaR4} and \emph{btaR5} acted as repressors on lipase production \cite{ulrich2004}. \emph{btaI1} and \emph{btaI3} inhibit lipase production while \emph{btaI2} enhances it. Even though the data in regard to lipase activation and repression implies that \emph{luxR123} and \emph{luxI123} represent complete circuit systems, the evidence was not compelling to Ulrich et al. to draw that conclusion without further investigation \cite{ulrich2004}.
It is unclear whether the studies by Conway et al. and Ulrich et al. refer to the same QS systems, if \emph{bviRI} and \emph{btaRI} are synonymous. If this is the case, a possible explanation for Conway et al. not detecting any impact of the \emph{bviR} mutant on lipase production could be given by them investigating what Ulrich et al. termed \emph{btaR2}, in which they also reported no discernible impact on lipase production. 
The complicated and cross-regulatory system surrounding lipase expression within the \emph{Burkholderia} genus may indicate the complexity of lipase production in other proteobacteria, including the role that QS plays to influence the expression.
\\
\\ \emph{\underline{Pseudomonas spp.}}
\\ \emph{P. aeruginosa} has two AHL dependent QS systems: \emph{las} and \emph{rhl} which each represent \emph{lux} homologs and respond to separate AHLs. The LasR:AHL complex controls \emph{rhlIR} expression, while RhlR activates \emph{lipA} transcription. Hence RhlR directly influences lipase production while LasR's influence is indirect. However a third non-AHL QS system, called the P. aeruginosa quinilone signal, regulates the other two systems \cite{juhas2005}. 


The AHLs are also influenced by other factors such as GacA, which promotes the QS regulatory cascade \cite{reimmann1997}, as well as RsmA which degrades AHLs \cite{pessi2001}. However even though RsmA degrades the QS signalling molecule, it positively affects lipase production as it binds to the \textit{lip} mRNA and stabilises the transcript to reduce the transcript's rate of degradation. RsmA in turn is regulated by RsmZ, whose suppression of RsmA therefore negatively affects lipase production \cite{heurlier2004}. 
\\
\\ \emph{\underline{Xenorhabdus nematophilus}} 
\\The insect pathogen \emph{Xenorhabdus nematophilus} resides in the intestinal tract of the symbiotic host \emph{Steinernema carpocapsae}. It aids the host in reproducing while the nematode represents a reservoir for the bacterium to infect insects \cite{herbert2007}. Lipase production is upregulated by  by the QS machinery \cite{dunphy_97} if \emph{flhDC}, which is the flagella master operon, is also expressed \cite{rosenau2000}. Another regulator is Lrp, which controls LrhA, and positively impacts lipase production and motility \cite{richards2008} 
The link between lipase expression, motility and virulence is established \cite{givaudan_00}, which suggests that the lipase contributes to the feeding mechanism while infecting rather than being a virulent agent itself \cite{richards2010}. Considering the organism's life style, which involves the necessity to leave the host in order to access the target, the controls associated with lipase expression are consistent with the requirement to move to attack the food source before releasing extracellular enzymes.
\\
\\ \emph{\underline{Serratia spp.}} 
\\Lipase production in both \emph{S. marcescens} and emph{S. proteamaculanss} is influenced by QS \cite{horng2002,shibatani2000,christensen_03}. Furthermore, in \emph{Serratia marcescens} surface attachment and biofilm formation in  is influenced by QS \cite{labbate2007} and all contribute to the virulent phenotype \cite{hejazi_97}.

\subsection{Aims}

This project explores the hypotheses that 
\begin{itemize}
\item Glycerol trioleate serves as a surface substratum for activated sludge floc formation;
\item Acetylated homoserine lactone mediated gene expression regulates lipase production in activated sludge flocs.
\end{itemize}

\noindent
Specifically we aim to

\begin{enumerate}
\item Investigate lipid morphology in activated sludge;
\item Monitor change in floc community in the presence of lipids;
%\item quantify to which extent AHLs partition into lipids;
\item Isolate and identify AHL producing bacteria degrading lipid in activated sludge.
\end{enumerate}
To this end the model lipid GT was incubated in activated sludge.



\bibliographystyle{acm}
\bibliography{thesis_ref}
\end{document}
