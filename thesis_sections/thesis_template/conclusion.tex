Bacterial floc formation is essential for the remediation of domestic wastewater. This study has sought to supplement the limited information available about floc formation and establishing the impact that lipid addition has on activated sludge morphology as well as microbial community structure. Deciphering the nature of colonisation in a biofilm-like manner around lipid droplets would extend our understanding of the mechanism of remediation. Droplets could be fixed cryogenically and sectioned before microscopic analysis. Monitoring the change of colonisation over time could aid in establishing the floc life cycle based on lipid colonisation and consumption. \\


Determining the exact concentration of lipids which elicits droplet formation and hence decrease floc settleability is essential for potentially monitoring wastewater influent and pre-empting failure of the remediation process. An experimental system to analyse the microbiome change with extended exposure to lipids was established. Future research should focus on 1 \% GT addition to mimic sludge behaviour in treatment plants. Highly abundant organisms which were affected by lipid enrichment were analysed with 16S rRNA DGGE, followed by sequencing and characterisation. Approximately 92 \% of the identified microbes belonged to the phylum proteobacteria, which is well known for using AHL mediated gene expression, including for lipase expression. Describing AHL mediated gene expression systems in isolates and assessment whether they regulate lipase activity should be conducted, possibly by \emph{luxI} knockout mutant construction.\\

Lipid droplet colonising isolates were screened with a LuxR biosensor for AHL production where 3 out of 5 isolates elicited a response from the biosensor. These isolates showed a higher AHL production when cultivated in the presence of lipid than without and the nature of the pink piggment produced by \emph{Pandoraea} sp. cultured with lipids should be investigated.

 in activated sludge fraction of wastewater influent is remediated and needs to be investigated further as lipid droplets could seed floc formation. Deciphering if biomass is within lipid droplets or if they attach at the lipid water interface is essential for 

 cells to droplets and whether cells are burrowed in lipid or attached to surface should be conducted. D

1) Trying lower lipid concentrations - closer to what is in activated sludge.

3) Growing isolates on lipids in solution and monitoring for floc formation - 
4) .
5) Knock out AHL syntheses and see how lipid based floc formation differs compared to wild type.
6) pink pigments