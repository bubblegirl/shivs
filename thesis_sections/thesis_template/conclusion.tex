Bacterial communication and coordination is essential for floc formation and wastewater remediation. The colonisation in a biofilm-like manner around lipid droplets extends our understanding about how the lipid fraction of wastewater influent is utilised for remediation and needs to be investigated further as lipid droplets could seed floc formation. SEM microscopy to showcase biofilm like attachment of cells to droplets and whether cells are burrowed in lipid or attached to surface should be conducted. Determining the exact concentration of lipids which elicits droplet formation and hence decrease floc settleability is essential for potentially monitoring wastewater influent and pre-empting failure of the remediation process. An experimental system to analyse the microbiome change with extended exposure to lipids was established. Highly abundant organisms which were affected by lipid enrichment were analysed with 16S rRNA DGGE, followed by sequencing and characterisation. Approximately 92 \% of the identified microbes belonged to the phylum proteobacteria, which is well known for using AHL mediated gene expression, including for lipase expression. Lipid droplet colonising isolates were screened with a LuxR biosensor for AHL production where 3 out of 5 isolates elicited a response from the biosensor. These isolates showed a higher AHL production when cultivated in the presence of lipid than without.


1) Trying lower lipid concentrations - closer to what is in activated sludge.
2) Deciphering if biomass is within lipid droplets or just outside. I don't think SEM will help here because it requires solvent washes for dehydration which will dissolve your lipids. Probably better to fix them cryogenically and section before light or fluorescence microscopy.
3) Growing isolates on lipids in solution and monitoring for floc formation - describe floc life cycle based on lipid colonisation and consumption.
4) Describing AHL mediated gene expression systems in isolates and assess if they regulate lipase activity.
5) Knock out AHL syntheses and see how lipid based floc formation differs compared to wild type.
6) pink pigments