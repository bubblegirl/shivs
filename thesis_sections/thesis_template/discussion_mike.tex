\documentclass[11pt]{article}
%\documentclass[draft]{article} for better debugging!
\usepackage[square,sort,comma,numbers]{natbib}

\linespread{1.3}

\addtolength{\textwidth}{2cm}
\addtolength{\hoffset}{-1cm}


\addtolength{\textheight}{2cm}
\addtolength{\voffset}{-1cm}

\begin{document}

The function of our society depends on safe and efficient municipal wastewater remediation while sustainably recycling water and reducing environmental impact. Waste influent created by domestic households is extensive and fluctuates in composition. Hence the wastewater treatment system for influent remediation needs to be adaptable to deal with the volume and variability. The biological component employed by the treatment facilities is the least well understood aspect. This study sought to develop an experimental system to explore the interaction between lipids and activated sludge. Specifically, five aims were addressed relating to spatial aspects of lipid partitioning and colonization, impacts on bacterial community composition, isolation of lipid degrading bacteria and testing for AHL production.

\subsection{Physical behaviour of lipids in activated sludge}
The physical partitioning behaviour of lipids upon introduction to activated sludge is poorly described. As a starting point it was unclear whether the addition of glycerol trioleate (GT) to activated sludge would result in the formation of a single free phase lipid droplet or whether sludge flocs and the incubation vessel would be coated in lipid with an absence of free phase or if multiple small lipid droplets or an emulsion would form. This lack of information on the physical partitioning behaviour in activated sludge hampers development of an experimental model for lipid dependent floc formation. To this end, 1 and 2 \% (v/v) GT was incubated in the presence of activated sludge and direct observations were made (Aim 1). \\

During the enrichment on 1 \% GT (v/v), multiple lipid droplets formed which were suspended throughout the sludge. Objects that resemble droplets in lipid challenged activated sludge have been reported previously (ref), however the lipid concentration to elicit this response was not described. It stands to reason that such lipid droplets represent an attractive surface substratum for colonization by activated sludge bacteria enabling evasion of predation and ready access to a carbon and energy source. Biofilm formation on lipid droplets would ultimately generate aggregated biomass resembling a floc. The result was encouraging in that it suggested lipid colonization may represent a realistic model system for floc formation and development. \\

When the GT concentration was increased to 2 \% (v/v), lipid droplets did not form and the sludge increased in viscosity and lightened in colour, resembling an emulsion. This is likely due to the excessive release of fatty acids released from the high concentration of triacylglyceride applied. The 2 \% (v/v) lipid concentration applied to the activated sludge was adapted from a study conducted by Haba et al \cite{haba2000isolation} on isolation of lipase producing bacteria. However the typical lipid concentration found in municipal wastewater is approximately 0.01 \% (v/v) \cite{Forster_92}. \\

Continued enrichment of the emulsion through sub-culturing resulted in increased viscosity to an almost solid state and white appearance in the tertiary sub-culture 45 days. No incidence of an emulsion like phenotype in the wastewater treatment plant setting has been found. Hence, these phenomena are considered to be artifacts of the unrealistically high GT concentration applied.

\subsection{Bacteria colonise lipid droplets in activated sludge}
Addition of 1 \% (v/v) GT to activated sludge resulted in lipid droplet formation. Epifluorescence microscopy was then used to observe physical interactions (spatial relationships) between the activated sludge biomass and the lipid droplets (Aim 2). \\

The microscopy images obtained clearly showcase the close association of biomass with the lipids. Whilst the aggregation around the droplets resembles biofilm formation, it is not clear if the physicochemical properties of cells and lipid droplets drive the association or if active energy dependent processes result in microbial colonization of the lipid droplets. \\

Whether the dense mass of cells observed is surrounding and growing on the lipid or whether it was embedded within the lipid phase is unclear from the microscopy images generated. Cells embedded in the lipid, with no access to water, are unlikely to proliferate. If lipids seed flocs, the buoyancy of the lipid core could be responsible for lack of settlement rather than the presence of lipids preventing flocculation as suggested by Forster et al \cite{Forster_92}. Determining which microbial communities dominate during lipid induced floc settlement failure could provide another aspect for monitoring and anticipation of system failure.

\subsection{Activated sludge bacteria produce lipases and degrade lipid droplets}

Lipid droplet size was observed to decrease over time until they were no longer visible. This is consistent with microbes colonising the droplets and degrading the surface substratum as a carbon and energy source. Lipase activity was also observed to increase in activated sludge samples amended with GT. An increase in lipase activity was observed in the EPS over the first 3 days of incubation but the greatest fold increase in lipase activity was observed in the culture supernatant between 3 and 6 days of incubation as the EPS based lipase activity decreased. It has previously been shown that lipases are weakly bound to the EPS of flocs through hydrogen bonding \cite{mayer1999role,wicker1987}. The sequential appearance of lipase first in the EPS and then in the supernatant fraction is likely due to release of lipases from the EPS.\\

While lipase activity was observed in the biomass fraction, there was little difference in activity when comparing GT amended treatments and unamended controls. The unamended controls displayed a high background level of lipase activity presumably owing to the presence of background levels of lipid in activated sludge. The decline in lipase activity observed in the EPS fraction in unamended controls is concordant with this, with the background lipid component being consumed. It should also be noted that as the viscosity of samples increased it was difficult to separate the three distinct fractions.

\subsection{Sphingomonads dominate in GT amended activated sludge.}
It is not known which bacteria are responsible for the degradation of lipids in activated sludge. This information is crucial in the development of an activated sludge floc formation model based on lipid colonization. As a starting point, the response of bacteria in activated sludge to the addition of GT was assessed using DGGE (Aim 3). It is evident from the DGGE community analysis that lipids have a profound impact on the microbial community structure in activated sludge. Whilst bacterial lineages enriched in the presence of GT can be implicated in lipid degradation in activated sludge this does not represent unambiguous evidence of lipid biodegradation by these bacteria.\\

Based on band intensity the most abundant bacterial lineages in the untreated sludge samples belonged to the \emph{Bacillus}, \emph{Dechloromonas} and \emph{Azospira} genera. \emph{Bacillus} species have been observed in sludge previously and the sequence retrieved here was associated with lipolytic activity in olive mill wastewater \cite{ertuugrul2007isolation}. Bacillus species have also been associated with bioflocculation in starch wastewater treatment \cite{deng2003characteristics}.\\

Dechloromonas species were observed in membrane fouling biofilms of municipal wastewater treatment plants in Japan, with over 30 \% of clones belonging to this genus \cite{miura2007membrane}. Azospira are also common activated sludge occupants \cite{tan2003dechlorosoma,reinhold2000reassessment,hunter2007azospira,wilhelmus2013microbiological}.
Overall, the DGGE profiles of the unamended sludge community was typical of activated sludge and did not differ greatly over the incubation period.\\

In sludge samples exposed to GT the DGGE profiles over time shifted dramatically. The most abundant bacteria observed in the unamended sludge were replaced within 9 days of incubation with \emph{Sphingomonas}, \emph{Novosphingobium} and \emph{Roseomonas} species.
\emph{Sphingomonas} species are present at about 5 - 10 \% relative abundance in sludge as shown by FISH \cite{neef1999detection}, and play an important role in wastewater remediation. Members of this genera degrade testosterone and sterol hormones as well as the pollutant nonylphenol \cite{fujii2001sphingomonas,roh201017beta}. \emph{Novosphingobium} species have been shown to play an important role in wastewater remediation by degrading toxic dyes and estrogen \cite{addison2007novosphingobium,hashimoto2009contribution}.
Roseomonas species have been found at about 5 \% relative abundance in activated sludge and are known to degrade organophosphate pesticides \cite{jiang2008bacterial,jiang2006isolation}. None of these lineages have previously been associated with lipid consumption in wastewater treatment plants. From the data generated here bacteria belonging to the \emph{Sphingomonas}, \emph{Novosphingobium} and \emph{Roseomonas} genera can be considered candidates for inclusion in an experimental system for investigating lipid based floc formation. \\


The emulsions contained an unrealistically high lipid concentration, hence is ideal as a model system to monitor the community change over long term GT exposure.
Tertiary enrichment samples were separated to monitor the community in an aggregate, the cells dispersed into the supernatant by the aggregate and the emulsion. \\



\emph{Sphingomonas} species were identified for 3 out of the 6 main bacterial lineages in the tertiary enrichments. Bands 2 and 10 from the aggregate supernatant, \emph{Sphingomonas} sp., is on the same position in the gel as band 13 for the t3 emulsion but returned as \emph{Stakelama pacifica}.
NCBI assigned the latter designation with and E-value of 3e$^{-77}$ from a 163 bp sequence, while the designation for \emph{Sphingomonas} sp. was with an E-value of 0.0, based on a 445 bp sequence from band 2. The two sequences were aligned with NCBI BLAST to investigate sequence similarity. The sequence alignment of band 13, which covers the 269 to 431 bp region of the amplified 16S rRNA fragment, aligns from the 269$^{th}$ bp of the \emph{Sphingomonas} with a sequence identity was 100 \% with an E-value of 7e$^{-87}$. It is concluded that band 13 represents the \emph{Sphingomonas} rather than \emph{Stakelama pacifica}.
The high variability of \emph{Sphingomonas} GC content is a common observation in DGGE \cite{qiao2012effect}.\\


Members of the \emph{Bradyrhizobium} genus are slow growing \cite{rebah2002wastewater} and usually associated with plant nodules. In plant symbiosis, they contribute by fixing nitrogen - an attribute required for successful activated sludge mediated remediation. When exposed to activated sludge, these microbes have been shown to become highly antibiotic and heavy metal resistant \cite{ahmad17samiullah}.\\


\emph{Oleomonas sagarensis} becomes prevalent in the emulsion but not in the aggregate or it's supernatant. This species is involved in breaking down urea \cite{kanamori2005allophanate,kanamori2004enzymatic} which is essential for nitrification of ammonia in activated sludge.
\emph{Rhodovarius lipocyclicus} was abundant in the aggregate supernatant and emulsion t2, but disappeared by the t3 sampling point. Information availible about this species is scarce \cite{kampfer2004rhodovarius} and it has not been previously implicated in ipid degradation. It could be of interest when dealing with microbiomes subject to extended lipid exposure. \\

While the majority of the microbes identified in the further enrichments were not found in the enrichment or control cultures for the 25 day duration of the enrichment experiment, these microorganisms are present in low abundance and became abundant during the next 56 days that the further enrichments ran for. Also the detection of organisms was limited as several bands were not sequenceable.\\


Hesham et al compared the microbial communities of two differently operated municipal wastewater treatment plants over six months via DGGE. The 11 OTUs common to both plants, and the most abundant, were 18 \% to alpha-proteobacteria and 18 \% to beta-proteobacteria \cite{Hesham_11}.


Wagner et al suggest FISH with 16S rRNA group-specific probes is necessary to accompany 16S rRNA sequencing sampling to ensure the accuracy of the OTU representation. However as this is a time consuming process, the majority of the community analysis is on 16S rRNA only \cite{Wagner_02} . A skewered representation of the operational taxonomic units in the community can arise from 16S primer bias which should be checked and accounted for.


\subsection{Lipid colonising isolates behave differently in the presence of lipids}

Directly from lipid droplets, 5 isolates were screened for AHL activity and 4 isolates were sequenced. All 4 identified isolates belong to the proteobacteria, which are known for AHL mediated gene expression of lipase. However despite their response to the LuxR bioassay, these results were not conclusive when trying to establish LuxI homologs from the NCBI database. Only \emph{Pseudomonas} species have had their lipase activity definitively linked to AHL mediated gene expression. \\

\emph{Achromobacter} and \emph{Enterobacter} elicited a markedly higher LuxR response when cultured in the presence of lipid, while \emph{Pseudomonas} only caused a response when cultured in the presence of lipids. The presence of lipids may impact the growth rate or it could directly impact the AHL profile produced.

Despite not activating the bioassay, a \emph{Pandoraea} sp. soil isolate has been shown to secrete the AHL \emph{N}-octanoylhomoserine lactone, which has 8 Cs attached to the lactone ring \cite{han2013pandoraea}. A search of NCBI Protein BLAST returns a LuxR homolog but not a LuxI homolog. The \emph{Achromobacter} sp. genome was searched a \emph{luxI} homolog by using a \emph{Burkholderia} sp. homolog as  this is a \emph{$\beta$}- proteobacterium known to contain an AHL synthase. The databases UniProt and SwissProt were used through EXPASY in order to search annotated genomes only and exclude draft genomes. This returned  22 \% similarity to an \emph{Agrobacterium} sp. LuxI homolog.

Isolate LC5 was of interest due to it's high LuxR activation, in either culture condition. The organism could not be identified as the sequencing data was inconclusive for both attempts.\\

Interestingly LC1, \emph{Pandoraea} sp., produces a pink phenotype when cultured in the presence of lipids but remains white in the absence. This could be due to pigment production whose expression pattern is linked to the presence of lipids. The nature of the pink phenotype and how it relates to lipid degradation should be investigated. 
% ref bioassays
Furthermore characterisation with bioassays targeting a range of AHLs should be conducted, such as \emph{Chromobacterium violacein} CV026 and \emph{Agrobacterium tumefaciens} A136.
Nonetheless a \emph{Pandoraea} sp.  As the activation was negligible in this study, this AHL may be out of the detection range of the LuxR bioassay and should be tested with the  bioassay. \\

\bibliographystyle{acm}
\bibliography{thesis_ref}
\end{document}