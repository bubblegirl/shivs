\documentclass[11pt]{article}
%Gummi|065|=)
\title{\textbf{Quorum-sensing influence on extracellular lipase prosuction in activated sludge}}
\author{Anna Liza Kretzschmar\\
        z3218219\\
        Supervisor: Mike Manefield}
\date{}

\usepackage{natbib}

\begin{document}
Treatment of wastewater is essential for the function of our civilisation. The biological aspect of wastewater treatment plants employs microbial flocs to oxidise organic compounds in the waste influent, specifically in activated sludge. This is combined with chemical and physical remediation strategies to yield remediated effluent that can be released into the environment without posing a pathological or environmental threat. 
It is established that municipal wastewater activated sludge comprises a core microbiome that is adaptable to deal with fluctuating wastewater influent. When the influent contains a high lipid influent, the flocculation process is prevented and the biological aspect of wastewater remediation stalls. The mechanism underlying this phenomenon is not known. This study seeks to establish an experimental system to further the understanding of this knowledge gap to breach the gap to preventing system failure.
The change in microbial composition due to extended lipid exposure is assessed using denaturing gradient gel electrophoresis as well as change in lipase expression. Furthermore lipid colonising microbes are isolated and screened for \emph{N}-acyl-L-homoserine lactone activity to assess a link with lipase expression. 
\end{document}