\documentclass[11pt]{article}
%\documentclass[draft]{article} for better debugging!

\usepackage{cite}

\linespread{1.3}

\addtolength{\textwidth}{2cm}
\addtolength{\hoffset}{-1cm}


\addtolength{\textheight}{2cm}
\addtolength{\voffset}{-1cm}

\begin{document}
Bacterial communication and coordination is essential for floc formation and wastewater remediation. Furthermore the colonisation in a biofilm-like manner around lipid droplets extends our understanding about how the lipid fraction of wastewater influent is utilised for remediation and needs to be investigated further. To analyse the microbes which participate in this process, an enrichment experiment implementing the protocols established is proposed as well as 16S rRNA analysis of samples via DGGE, followed by sequencing and characterisation. 

Determining the exact concentration of lipids which elicits droplet formation and hence decrease floc settleability is essential for potentially monitoring wastewater influent and pre-empting failure of the remediation process. 

SEM microscopy to showcase bioflm like attachment on droplets and whether cells are burrowed in lipid or attached to surface

\end{document}