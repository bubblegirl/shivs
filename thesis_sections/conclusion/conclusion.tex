\documentclass[11pt]{article}
%\documentclass[draft]{article} for better debugging!

\usepackage{cite}

\linespread{1.3}

\addtolength{\textwidth}{2cm}
\addtolength{\hoffset}{-1cm}


\addtolength{\textheight}{2cm}
\addtolength{\voffset}{-1cm}

\begin{document}
Bacterial communication and coordination is essential for floc formation and wastewater remediation. The colonisation in a biofilm-like manner around lipid droplets extends our understanding about how the lipid fraction of wastewater influent is utilised for remediation and needs to be investigated further as lipid droplets could seed floc formation. SEM microscopy to showcase bioflm like attachment of cells to droplets and whether cells are burrowed in lipid or attached to surface should be conducted. Determining the exact concentration of lipids which elicits droplet formation and hence decrease floc settleability is essential for potentially monitoring wastewater influent and pre-empting failure of the remediation process. An experimental system to analyse the microbiome change with extended exposure to lipids was established. Highly abundant organisms which were affected by lipid enrichment were analysed with 16S rRNA DGGE, followed by sequencing and characterisation. Approximately 92 \% of the identified microbes belonged to the phylum proteobacteria, which is well known for using AHL mediated gene expression, including for lipase expression. Lipid droplet colonising isolates were screened with a LuxR biosensor for AHL production where 3 out of 5 isolates elicited a response from the biosensor. These isolates showed a higher AHL production when cultivated in the presence of lipid than without.
\end{document}