\documentclass[11pt]{article}
%\documentclass[draft]{article} for better debugging!
\usepackage{placeins}
\usepackage{cite}
\usepackage{graphicx}
\usepackage{rotating}
%download float barrier sty - uni mail
\linespread{1.3}

\addtolength{\textwidth}{2cm}
\addtolength{\hoffset}{-1cm}


\addtolength{\textheight}{2cm}
\addtolength{\voffset}{-1cm}

\begin{document}


\subsection{Bacteria colonise lipids in activated sludge}
It was essential to establish an experimental system and adapt protocols to observe the impact of lipids on activated sludge before a controlled enrichment experiment could be conducted. 

\subsubsection{\emph{change in morphology}}
A trial run preceded the actual enrichment run. The trial was conducted with 1 \% GT  addition to sludge replicates and sampling frequency was every 7 days as the expectation was that the enrichment would run for an extended period of time. GT formed a transparent and hydrophobic layer on the surface of the enrichment cultures, which was replaced with white droplets suspended below the culture surface within 7 days (Figure XB). While varying in size, the droplets consistently decreased in diameter with increased incubation period until the they were indistinguishable from the sludge after 21 days. Droplet formation decreased floc settlement for the duration of their presence. 
Due to the lipid being assimilated into the sludge within the first 3 sampling points, the sampling frequency was increased from 7 to every 3 days. To compensate, the GT \% was increased from 1 to 2 with the effect of droplet formation no longer taking place. Instead the cultures became increasingly viscous and lighter in colour (Figure XA). 
\begin{figure}
\includegraphics[scale=.77]{cultures.jpg}
\caption{Impact of lipid addition to activated sludge cultures; A) the comparison between 2 \% GT enrichment and no treatment; B) lipid droplet formation in the presence of 1 \% GT in trial run; C) the lack of settleability of sludge after incubation with 2 \% GT; D) secondary enrichment by transferral of aggregate into sludge supernatant and GT.}
\end{figure}
\FloatBarrier

\subsubsection{\emph{Microscopy}}
 Droplets were isolated and stained with SybrGreen before imaging with epifluorescence microscopy (Figure xx). The fluorophore binds to DNA and emits green light upon excitation while lipid appears yellow and water appears black.

\begin{figure}
\includegraphics[scale=.4]{august_microscopy.png}
\caption{Imaging of lipid droplets post SybrGreen staining with fluorescence microscopy. A); B); C); D); E) and F) demonstrate the close association of biomass with lipid compared to the intermediate liquid phase.}
\end{figure}


Biomass association with lipid droplets was evident (Figure cc). As lipid droplets were washed twice before staining and imaging, biomass was firmly attached to the droplets.
\FloatBarrier
\subsubsection{\emph{Monitoring lipase production}}
Lipase activity was detected upon colorimetric change from p-nitrophenyl palmitate cleavage in the substrate solution. The assay separated  EC, EPS bound and cell membrane bound lipases from each enrichment and Figure zz shows the total average lipase activity, with the enriched cultures displayed a higher activity than the control replicates.
                                                                                                                                                                                                                                         

\begin{figure}
\includegraphics[scale=1]{lip_av.PNG}
\caption{Total lipase activity of all fractions and replicates}
\end{figure}

The lipase activity in the supernatant fraction of the enriched cultures spikes between 3 and 9 days as well as 15 to 18 days in comparison to the untreated activate sludge where the lipase activity remains at a consistent level(Figure gg).

\begin{figure}
\includegraphics[scale=1.1]{lip_sn.PNG}
\caption{Average lipase activity in the supernatant fraction}
\end{figure}

An immediate increase of lypolytic activity in the EPS  fraction within the first 3 days is evident from Figure rr. Within 9 days the lipase activity in this fraction drops below the initial activity. T this time point the activty within the control cultures exceeds the enrichment cultures, which decreases rapidly and remains below the enrichment cultures lipase activity for all other time points.

\begin{figure}
\includegraphics[scale=1.1]{lip_eps.PNG}
\caption{Average lipase activity in the EPS fraction}
\end{figure}

It is evident from  Figure xx that membrane bound lipase is the most abundant across both control and enrichment cultures, where the average activity in the control replicates increased in comparison to the enriched cultures between days 9 to 22.

\begin{figure}
\includegraphics[scale=1.1]{lip_biom.PNG}
\caption{Average lipase activity in the biomass fraction}
\end{figure}
\FloatBarrier
\subsection{Lipid colonising isolates behave differently in the presence of lipids}
% change in morphology in LC1 and LuxR data

The isolates LC1 - LC5 were from lipid droplets in 1 \% GT enriched activated sludge. These were cultured on a minimal medium, with and without the presence of lipids. LC1 cultured in the presence of lipid displays a pink phenotype (Figure VA) and appears white when cultured in the absence of lipids (Figure VB).
This further led to the question whether AHL profiles are affected by these culture conditions as well as morphology. This was addressed by using LuxR bioassay as recorded in Figure ll. LC1 exhibits minimal AHL activity detectable by LuxR, while LC4 displays no activity. LC5 elicits the highest response form the biosensor, marginally more so when cultured in the presence of lipids. LC2 and LC3 both evoke a higher LuxR responses when cultured in the presence of lipid than without.

\begin{figure}
\includegraphics[scale=.65]{LC1_comp.png}
\caption{Lipid colonising isolate LC1 cultured in the presence (A) and absence (B) of lipid in LCM medium}
\end{figure}

\begin{figure}
\includegraphics[scale=1.1]{LCM_LuxR.png}
\caption{Response to LuxR bioassay by lipid colonising isolates, comparatively cultured in the presence and absence of lipid }
\end{figure}

Sequencing of lipid colonising isolates resulted in identification of LC1 - LC4 with NCBI BLAST (Table cc), while LC5 was not identifiable as only 6.7 \% of the bases in the chromatogram of high enough quality to compare to the NCBI BLAST database.
	
\begin{table}
\caption{Sequencing results for 16S rRNA fragments of lipid colonising isolates 1 - 4, the sequences used for identification can be found in Supplementary Material  x.}
\begin{tabular}{ | l | p{7.8cm} | p{3cm} | l | }
\hline
Isolate & Bacteria with highest identity \& Acc. no. & Class & E-value \\
\hline
1 &  \emph{Pandoraea} sp. (KF378759.1) & \emph{$\beta$-proteobacteria} & 2e$^{-79}$ \\
\hline
2 & Uncultured \emph{Achromobacter} sp. (KF448091.1) & \emph{$\beta$-proteobacteria} & 1e$^{-139}$ \\
\hline
3 & \emph{Enterobacter} sp. (KF411353.1) & \emph{$\gamma$-proteobacteria} & 0.0 \\
\hline
4 & \emph{Pseudomonas} sp. (KC822768.1) & \emph{$\gamma$-proteobacteria} & 0.0 \\
\hline
\end{tabular}
\end{table}
\FloatBarrier
\subsection{Micobial community changes in the presence of lipids}
The PCR products of 16S DGGE PCR  of replicate 4 were run on DGGE (Figure dd) where the numbered bands excised and successfully amplified and sequenced (Table gg). 

\begin{figure}
\includegraphics[scale=2.7]{DGGE_R4_450bp_thesis.jpg}
\caption{Community structure of control and enriched cultures of replicate 4, from day 0 to day 25 (D0 - D25).}
\end{figure}

\begin{table}
\caption{Sequencing results for 16S rRNA DGGE fragments of replicate 4, the sequences used for identification with NCBI BLAST can be found in Supplementary Material  x.}
\begin{tabular}{ | l | p{7.8cm} | p{3cm} | l | }
\hline
DGGE band & Bacteria with highest identity \& Acc. no. & Class & E-value \\
\hline
1   &  \emph{Bacteroidetes} (JX473581.1) & --- & 8e$^{-45}$ \\
\hline
2  & \emph{Bacillus} sp. (GU271888.1) & \emph{Bacilli} & 5e$^{-70}$ \\
\hline
3 & Uncultured \emph{Dechloromonas} sp. (JQ012310.1) & \emph{$\beta$-proteobacteria} & 0.0 \\
\hline
4 & \emph{Azospira oryzae} (KF260987.1) & \emph{$\beta$-proteobacteria} & 1E$^{-50}$ \\
\hline
5 & Uncultured \emph{Xanthmonadales} (KC588330.1) & \emph{$\gamma$-proteobacteria} & 0.0 \\
\hline
6 & Uncultured \emph{Dechloromonas} sp. (KF003189.1) & \emph{$\beta$-proteobacteria} & 4e$^{-103}$ \\
\hline
7 & \emph{Novosphingobium} sp. (KF544940.1) & \emph{$\alpha$-proteobacteria} & 1e$^{-173}$ \\
\hline
8 & \emph{Novosphingobium} (KF544932.1) & \emph{$\alpha$-proteobacteria} & 4e$^{-61}$ \\
\hline
9 & \emph{Sphingomonas} sp. (AY521009.2) & \emph{$\alpha$-proteobacteria} & 0.0 \\
\hline
10 & \emph{Sphingomonas suberifaciens} (AY521009.2) & \emph{$\alpha$-proteobacteria} & 3e$^{-119}$ \\
\hline
11 & \emph{Sphingomonas} sp. (JQ928361.1) & \emph{$\alpha$-proteobacteria} & 5e$^{-86}$ \\
\hline
12 & \emph{Roseomonas} sp.  (KF254767.1) & \emph{$\alpha$-proteobacteria} & 5e$^{-65}$ \\
\hline
\end{tabular}

\end{table}

The PCR products of 16S DGGE PCR  of various further enrichments were run on DGGE (Figure dd) where the numbered bands excised and successfully amplified and sequenced (Table gg). t3 
\FloatBarrier
\begin{figure}
\includegraphics[scale=1.2]{DGGE_misc_450bp_thesis.jpg}
\caption{Community structure of aggregates and emulsions from various day points (D) as well as secondary (t2) and tertiary (t3) enrichments, which are uniformly taken after 45th day, and community comparison between 1 and 2 \% GT enrichment.}
\end{figure}

\begin{table}
\caption{Sequencing results for 16S rRNA DGGE fragments of various further enrichments, the sequences used for identification with NCBI BLAST can be found in Supplementary Material x.}
\begin{tabular}{ | l | p{7.8cm} | p{3cm} | l | }
\hline
DGGE band & Bacteria with highest identity \& Acc. no. & Class & E-value \\
\hline
1 & \emph{Nevskia} sp. (GQ845011.1) & \emph{$\gamma$-proteobacteria} & 0.0  \\
\hline
2 & \emph{Sphingomonas} sp. (KC172307.1) & \emph{$\alpha$-proteobacteria} & 0.0 \\
\hline
3 & \emph{Sphingobium} sp. (KF437579.1) & \emph{$\alpha$-proteobacteria} & 2e$^{-74}$ \\
\hline
4 & \emph{Sphingomonas} sp. (KF544924.1) & \emph{$\alpha$-proteobacteria} & 0.0  \\
\hline
5 & \emph{Rhodovarius lipocyclicus} (NR\_025629.1) & \emph{$\alpha$-proteobacteria} & 3e$^{-114}$ \\
\hline
6 & \emph{Xanthobacter} sp. (AB847934.1) & \emph{$\alpha$-proteobacteria} & 5e$^{-86}$  \\
\hline
7 & \emph{Xanthobacter} sp. (AB245351.1) & \emph{$\alpha$-proteobacteria} & 3e$^{-45}$  \\
\hline
8 & \emph{Sphingomonas} sp.(KF551133.1) & \emph{$\alpha$-proteobacteria} & 2e$^{-100}$  \\
\hline
9 & \emph{Bradyrhizobium} sp. (JX505076.1) & \emph{$\alpha$-proteobacteria} & 3e$^{-165}$  \\
\hline
10 & \emph{Sphingomonas} sp. (EF636068.1) & \emph{$\alpha$-proteobacteria} & 0.0  \\
\hline
11 & Candidatus \emph{Competibacter} sp. (JQ480426.1) & \emph{$\gamma$-proteobacteria} & 1e$^{-123}$  \\
\hline
12 & \emph{Sphingomonas} sp. (HE974351.1) & \emph{$\alpha$-proteobacteria} &  1e$^{-148}$ \\
\hline
13 & \emph{Stakelama pacifica} (HE662817.1) & \emph{$\alpha$-proteobacteria} & 3e$^{-77}$  \\
\hline
14 & \emph{Oleomonas sagaranensis} (AJ784808.1) & \emph{$\alpha$-proteobacteria} & 6e$^{-101}$  \\
\hline
\end{tabular}

\end{table}
\FloatBarrier
\end{document}