\documentclass[11pt]{article}
%\documentclass[draft]{article} for better debugging!

\usepackage{cite}
\usepackage{graphicx}
\linespread{1.3}

\addtolength{\textwidth}{2cm}
\addtolength{\hoffset}{-1cm}


\addtolength{\textheight}{2cm}
\addtolength{\voffset}{-1cm}

\begin{document}


\subsection{Bacteria colonise lipids in activated sludge}
On addition to AS GT formed a transparent and hydrophobic layer on the surface of the enrichment cultures. After 7 days the GT morphology changed to white droplets suspended in the culture below the surface. Whilst varied in size they consistently decreased in diameter with increased incubation period until the droplets were indistinguishable after 21 days. Droplet formation decreased floc settlement in all GT enriched cultures.

\subsubsection{\emph{change in morphology}}

\subsubsection{\emph{Microscopy}}
 Droplets were isolated and stained with SybrGreen before imaging with epifluorescence microscopy (\textbf{Figure 1}). 

\begin{figure}
\includegraphics[scale=.4]{august_microscopy.png}
\caption{Imaging of washed lipid droplets post SybrGreen staining via fluorescence microscopy. The dye stains DNA green, lipid appears yellow. \textit{A}) shows inorganic matter which appears orange; \textit{B}), \textit{C}) and \textit{D}) show bacterial attachment to lipid substrates \textit{E}) \textit{F}).}
\end{figure}


\textbf{Figure 1 } shows biomass attached to lipid droplets. As the lipid droplets were washed twice before staining and imaging, the biomass is firmly attached to the droplets. \textbf{Figure 1}\textit{B} and \textbf{Figure 1}\textit{D} show cells colonising the surface of a lipid droplet. \textbf{Figure 1}\textit{C} shows cells internal to the lipid droplet.

\subsubsection{\emph{Monitoring lipase production}}
Lipase was activity was detected upon colorimetric change from p-nitrophenyl palmitate cleavage in the substrate solution. The assay separated  EC, EPS bound and cell membrane bound lipases from each enrichment (\textbf{Graph 2}).


%\begin{figure}
%\includegraphics[scale=1]{lipase_activity.png}
%\caption{Shows the comparative average lipase activity after 8 and 15 days of incubation for the different lipases in each enrichment.}
%\end{figure}


It is evident from  \textbf{Graph 2} that membrane bound lipase is the most abundant across all enrichment cultures at both time points, where highest activity was recorded in no treatment, followed by the acetate enrichment and then GT enrichment.  The activity for the 8 day time point was close to synonymous with the 15 day time point, with highest difference of 0.1 seen in GT membrane lipase activity.

\subsection{Lipid colonising isolates behave differently in the presence of lipids}
% change in morphology in LC1 and LuxR data

\begin{figure}
\includegraphics[scale=0.9]{LCM_LuxR.png}
\caption{Wastewater processing as employed by St Mary's treatment plant \cite{stmarys}}
\end{figure}


\subsection{Micobial community changes in the presence of lipids}
%DGGE and sequencing data

\end{document}