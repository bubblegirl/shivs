\documentclass[11pt]{article}
%\documentclass[draft]{article} for better debugging!

\usepackage{cite}
\usepackage{graphicx}
\usepackage{rotating}
%download float barrier sty - uni mail
\linespread{1.3}

\addtolength{\textwidth}{2cm}
\addtolength{\hoffset}{-1cm}


\addtolength{\textheight}{2cm}
\addtolength{\voffset}{-1cm}

\begin{document}


\subsection{Bacteria colonise lipids in activated sludge}
It was essential to establish an experimental system and adapt protocols to observe the impact of lipids on activated sludge before a controlled enrichment experiment could be conducted. Hence a trial run preceded the actual enrichment run. The trial was conducted with 1 \% GT  addition to sludge replicates and sampling frequency was every 7 days as the expectation was that the enrichment would run for an extended period of time. GT formed a transparent and hydrophobic layer on the surface of the enrichment cultures, which was replaced with white droplets suspended below the culture surface within 7 days as can be seen in Figure XB. While varying in size, the droplets consistently decreased in diameter with increased incubation period until the they were indistinguishable from the sludge after 21 days. Droplet formation decreased floc settlement for the duration of their presence. 
Due to the lipid being assimilated into the sludge within the first 3 sampling points, the sampling frequency was increased from 7 to 3 days. To compensate, the GT \% was increased from 1 to 2 with the effect of droplet formation no longer taking place. Instead the cultures became increasingly viscous and lighter in colour as can be seen from Figure XA. 

\subsubsection{\emph{change in morphology}}
\begin{figure}
\includegraphics[scale=.9]{cultures.jpg}
\caption{Impact of lipid addition to activated sludge cultures; A) the comparison between 2 \% GT enrichment and no treatment; B) lipid droplet formation in the resence of 1 \% GT; C) the lack of settleability of sludge after incubation with 2 \% GT; D) secondary enrichment by transferral of aggregate into sludge supernatant and GT}
\end{figure}


\subsubsection{\emph{Microscopy}}
 Droplets were isolated and stained with SybrGreen before imaging with epifluorescence microscopy (\textbf{Figure 1}). 

\begin{figure}
\includegraphics[scale=.4]{august_microscopy.png}
\caption{Imaging of washed lipid droplets post SybrGreen staining via fluorescence microscopy. The dye stains DNA green, lipid appears yellow. \textit{A}) shows inorganic matter which appears orange; \textit{B}), \textit{C}) and \textit{D}) show bacterial attachment to lipid substrates \textit{E}) \textit{F}).}
\end{figure}


\textbf{Figure xx } shows biomass attached to lipid droplets. As the lipid droplets were washed twice before staining and imaging, the biomass is firmly attached to the droplets. \textbf{Figure 1}\textit{B} and \textbf{Figure 1}\textit{D} show cells colonising the surface of a lipid droplet. \textbf{Figure 1}\textit{C} shows cells internal to the lipid droplet.

\subsubsection{\emph{Monitoring lipase production}}
Lipase was activity was detected upon colorimetric change from p-nitrophenyl palmitate cleavage in the substrate solution. The assay separated  EC, EPS bound and cell membrane bound lipases from each enrichment (\textbf{Graph 2}).

\begin{figure}
\includegraphics[scale=1]{lip_av.PNG}
\caption{Total lipase activity of all fractions and replicates}
\end{figure}

\begin{figure}
\includegraphics[scale=1.1]{lip_sn.PNG}
\caption{Average lipase activity in the supernatant fraction}
\end{figure}

\begin{figure}
\includegraphics[scale=1.1]{lip_eps.PNG}
\caption{Average lipase activity in the EPS fraction}
\end{figure}

\begin{figure}
\includegraphics[scale=1.1]{lip_biom.PNG}
\caption{Average lipase activity in the biomass fraction}
\end{figure}

%\begin{figure}
%\includegraphics[scale=1]{lipase_activity.png}
%\caption{Shows the comparative average lipase activity after 8 and 15 days of incubation for the different lipases in each enrichment.}
%\end{figure}


It is evident from  Figure xx that membrane bound lipase is the most abundant across both control and enrichment cultures, where the average activity in the control replicates increased in comparison to the enriched cultures between days 9 to 22.

\subsection{Lipid colonising isolates behave differently in the presence of lipids}
% change in morphology in LC1 and LuxR data

\begin{figure}
\includegraphics[scale=.65]{LC1_comp.png}
\caption{Lipid colonising isolate LC1 cultured in the presence and abscence of lipid in LCM medium}
\end{figure}

\begin{figure}
\includegraphics[scale=1.1]{LCM_LuxR.png}
\caption{Response to LuxR bioassay by lipid colonising isolates, comparatively cultured in the presence and absence of lipid }
\end{figure}


\begin{sidewaystable}[!htbp]
\begin{tabular}{ | l | p{9cm} | p{4.5cm} | l | }
\hline
Isolate & Bacteria with highest identity \& Acc. no. & CLass & E-value \\
\hline
1 &  \emph{Pandoraea} sp. (KF378759.1) & \emph{$\beta$-proteobacteria} & 2e$^{-79}$ \\
\hline
2 & Uncultured \emph{Achromobacter} sp. (KF448091.1) & \emph{$\beta$-proteobacteria} & 1e$^{-139}$ \\
\hline
3 & \emph{Enterobacter} sp. (KF411353.1) & \emph{$\gamma$-proteobacteria} & 0.0 \\
\hline
4 & \emph{Pseudomonas} sp. (KC822768.1) & \emph{$\gamma$-proteobacteria} & 0.0 \\
\hline
\end{tabular}
\caption{Sequencing results for 16S rRNA fragments of lipid colonising isolates 1 - 4, the sequences used for identification can be found in Appendix x}
\end{sidewaystable}

\subsection{Micobial community changes in the presence of lipids}
%DGGE and sequencing data

\begin{figure}
\includegraphics[scale=2.7]{DGGE_R4_450bp_thesis.jpg}
\caption{Community structure of control and enriched cultures of replicate 4}
\end{figure}

\begin{sidewaystable}[!htbp]
\begin{tabular}{ | p{1.2cm} | p{10cm} | p{4cm} | p{2cm} | }
\hline
DGGE band & Bacteria with highest identity \& Acc. no. & Class & E-value \\
\hline
1   &  \emph{Bacteroidetes} (JX473581.1) &  & 8e$^{-45}$ \\
\hline
2  & \emph{Bacillus} sp. (GU271888.1) & \emph{Bacilli} & 5e$^{-70}$ \\
\hline
3 & Uncultured \emph{Dechloromonas} sp. (JQ012310.1) & \emph{$\beta$-proteobacteria} & 0.0 \\
\hline
4 & \emph{Azospira oryzae} (KF260987.1) & \emph{$\beta$-proteobacteria} & 1E$^{-50}$ \\
\hline
5 & Uncultured \emph{Xanthmonadales} (KC588330.1) & \emph{$\gamma$-proteobacteria} & 0.0 \\
\hline
6 & Uncultured \emph{Dechloromonas} sp. (KF003189.1) & \emph{$\beta$-proteobacteria} & 4e$^{-103}$ \\
\hline
7 & \emph{Novosphingobium} sp. (KF544940.1) & \emph{$\alpha$-proteobacteria} & 1e$^{-173}$ \\
\hline
8 & \emph{Novosphingobium} (KF544932.1) & \emph{$\alpha$-proteobacteria} & 4e$^{-61}$ \\
\hline
9 & \emph{Sphingomoas} sp. (AY521009.2) & \emph{$\alpha$-proteobacteria} & 0.0 \\
\hline
10 & \emph{Sphingomonas suberifaciens} (AY521009.2) & \emph{$\alpha$-proteobacteria} & 3e$^{-119}$ \\
\hline
11 & \emph{Sphingomonas} sp. (JQ928361.1) & \emph{$\alpha$-proteobacteria} & 5e$^{-86}$ \\
\hline
12 & \emph{Roseomonas} sp.  (KF254767.1) & \emph{$\alpha$-proteobacteria} & 5e$^{-65}$ \\
\hline
\end{tabular}
\caption{Sequencing results for 16S rRNA DGGE fragments of replicate 4, the sequences used for identification can be found in Appendix x}
\end{sidewaystable}

\begin{figure}
\includegraphics[scale=1.2]{DGGE_misc_450bp_thesis.jpg}
\caption{Community structure of aggregates, emulsions, secondary and tertiary enrichments}
\end{figure}

\begin{sidewaystable}[!htbp]
\begin{tabular}{ | l | p{9cm} | p{4.5cm} | l | }
\hline
DGGE band & Bacteria with highest identity \& Acc. no. & Class & E-value \\
\hline
1 & \emph{Nevskia} sp. (GQ845011.1) & \emph{$\gamma$-proteobacteria} & 0.0  \\
\hline
2 & \emph{Sphingomonas} sp. (KC172307.1) & \emph{$\alpha$-proteobacteria} & 0.0 \\
\hline
3 & \emph{Sphingobium} sp. (KF437579.1) & \emph{$\alpha$-proteobacteria} & 2e$^{-74}$ \\
\hline
4 & \emph{Sphingomonas} sp. (KF544924.1) & \emph{$\alpha$-proteobacteria} & 0.0  \\
\hline
5 & \emph{Rhodovarius lipocyclicus} (NR\_025629.1) & \emph{$\alpha$-proteobacteria} & 3e$^{-114}$ \\
\hline
6 & \emph{Xanthobacter} sp. (AB847934.1) & \emph{$\alpha$-proteobacteria} & 5e$^{-86}$  \\
\hline
7 & \emph{Xanthobacter} sp. (AB245351.1) & \emph{$\alpha$-proteobacteria} & 3e$^{-45}$  \\
\hline
8 & \emph{Sphingomonas} sp.(KF551133.1) & \emph{$\alpha$-proteobacteria} & 2e$^{-100}$  \\
\hline
9 & \emph{Bradyrhizobium} sp. (JX505076.1) & \emph{$\alpha$-proteobacteria} & 3e$^{-165}$  \\
\hline
10 & \emph{Sphingomonas} sp. (EF636068.1) & \emph{$\alpha$-proteobacteria} & 0.0  \\
\hline
11 & Candidatus \emph{Competibacter} sp. (JQ480426.1) & \emph{$\gamma$-proteobacteria} & 1e$^{-123}$  \\
\hline
12 & \emph{Sphingomonas} sp. (HE974351.1) & \emph{$\alpha$-proteobacteria} &  1e$^{-148}$ \\
\hline
13 & \emph{Stakelama pacifica} (HE662817.1) & \emph{$\alpha$-proteobacteria} & 3e$^{-77}$  \\
\hline
14 & \emph{Oleomonas sagaranensis} (AJ784808.1) & \emph{$\alpha$-proteobacteria} & 6e$^{-101}$  \\
\hline
\end{tabular}
\caption{Sequencing results for 16S rRNA DGGE fragments of various further enrichments, the sequences used for identification can be found in Appendix x}
\end{sidewaystable}

\end{document}