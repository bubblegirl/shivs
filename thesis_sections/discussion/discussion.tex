\documentclass[11pt]{article}
%\documentclass[draft]{article} for better debugging!

\usepackage{cite}

\linespread{1.3}

\addtolength{\textwidth}{2cm}
\addtolength{\hoffset}{-1cm}


\addtolength{\textheight}{2cm}
\addtolength{\voffset}{-1cm}

\begin{document}
\subsection{Bacteria colonise lipids in activated sludge}
%increase of GT from 1 to 2 \% due to incr sampling freq after trial - massive morphological change. needed to be investigated. also conc way too high to emulate actual setting but.. precedence form xx paper



%Oleic acid is the most common fatty acid found in native as well as used olive and sunflower oils, which are commonly used in domestic settings \cite{haba2000isolation}. 

The attachment of biomass to the droplets is evident in \textbf{Figure xx}, correlation with the decrease in lipid droplet size indicates that the microbes may colonise the droplets and break down the GT to consume the fatty acids as a carbon source. It follows that lipids could seed floc formation in activated sludge. This observation complements the highest lipase activity recorded for membrane bound lipases in \textbf{xx}.
% lipid conc way too high - compare the o.o1% thing with the paper that I got the 2% from

The lipase activity observed in the average total activity of the control comes from the membrane bound lipase fraction. Activated sludge contains a baseline lipid concentration which serves as a C source which could explain the maintenance of lipase for rapid degradation. Lipase activity in the EPS fraction almost doubles within the first 3 days, followed by steady decline while lipase activity in the supernatant fraction increases. Secreted lipase expressed by the \emph{$\gamma$}-proteobacterium \textit{Pseudomonas aeruginosa} associates with alginate within the EPS of flocs, which anchors the enzyme with weak bonding forces \cite{mayer1999role,wicker1987}. Hence the lipase  liberates fatty acids close to the cell. It follows that this mechanism is commonly employed and binding lipase to the EPS, and the delayed appearance of lipase in the supernatant fraction could be due to liberation from the hydrogen bonding and hence the floc association.
% link eps the sn lipase to floc life cycle?

The direct colonisation of the lipid droplets was unexpected as this behaviour in AS sludge has not been recorded in the literature. However it has been recorded by Forster et al that the presence of lipids decreases flocculation and sludge settlement \cite{Forster_92}, the latter of which was observed in this experiment. If lipids seed flocs, the behaviour of the lipid core could be responsible for lack of settlement rather than preventing flocculation. Hence it is essential to further investigate the flocs and which bacteria are actively involved in this vital aspect of wastewater remediation.

\subsubsection{change in morphology?}
droplet formation and emulsion dep on concentration - high lipid influent is kown to decrease settleability. however these concs are excessive (after which paper?) and much lower normally (check ref in intro). The viscosity of the 2 \% GT enriched cultures is a kin to an emulsion

\subsection{Microbial community change in the presence of lipids}
DGGE over pyrosequncing because it allows me to focus on the enriched bugs
yang 12 uses DGGE to track microbiome changes in ww!
wang 12 does pyroseq on diff wwtps in china. part of the dataset is initial COD! ompare to DGGE results. temp MOST IMPORTANT INFLUENCE shown by Wang 12, and within ref siggins and Wells.

However, as Wagner et al have suggested, fluorescence in situ hybridisation (FISH) with 16S rRNA  group-specific probes needs to accompany 16S rRNA sequencing sampling to ensure the accuracy of the OTU representation. However as this is a time consuming process, the majority of the community analysis is on 16S rRNA only \cite{Wagner_02} . A skewered representation of the OTUs in the community can arise from 16S primer bias which should be checked and accounted for.

The initial time points for the enrichment culture replicates did not yield DNA extracts that could be utilised for further analysis from days 3 and 6 for the controls, as well as day 3 for the enriched culture. This is most likely due to the pH of the phenols solution - initially it was pH 4. This pH is suitable for RNA extaction while phenol solution pH 8 is preferable for DNA extraction. The pH of the phenol solution was adjusted from day 9 onwards.

Hesham et al compared the microbial communities of two differently operated municipal WWT plants over six months via DGGE. They 11 OTUs common to both plants, and the most abundant, were 18 \% to alpha-proteobacteria and 18 \% to beta-proteobacteria. The study demonstrates the similarities between microbial communities in the WWT plants and the adequacy of utilising DGGE for comparison \cite{Hesham_11}.
% yeah.. compare to the rest of the studies in table as well/
% sphingomonas high GC varience paper


\emph{Nevskia} sp. are slow growing, where colony formation takes about a week, and they prodce lipase \cite{kim2011nevskia}. This genera has been isolated as the most efficient surfactant producer and hence lipid remediation in palm oil mill effluent by Chooklin et al \cite{hooklinutilization}.

\emph{Sphingomonas} sp. are present at about 5 - 10 \% in sludge as shown by FISH \cite{neef1999detection}, and play an important role in wastewater remediation. Members of this genera degrade testosterone and sterol hormones as well as the polutant nonylphenol \cite{fujii2001sphingomonas,roh201017beta}.

Limited information is availible about \emph{Rhodovarius lipocyclicus}, beyond the basic information required for classification \cite{kampfer2004rhodovarius}.

\emph{Xanthobacter} sp. remediate aliphatic halogenated compounds which are commonly found in municipal (?) wastewater \cite{janssen1985degradation} and have been shown to co-ordinate with \emph{Novosphingobium} sp. to degrade polyvinyl alcohol \cite{rong2009symbiotic}. \emph{Novosphingobium} sp. also degrade polycyclic aromatic hydrocarbons, which can by synthesised from lipid precursors \cite{addison2007novosphingobium}.

Members of the \emph{Bradyrhizobium} genera are slow growing \cite{rebah2002wastewater} and usually associated with plant nodules. In plant symbiosis, they cntribute by fxing nitrogen - an atribute required for successful A/A/O process. When exposed to activated sludge, these microbes have been shown to become highly antibiotic and heavy metal resistant.

Candidatus \emph{Competibacter} are glycogen accumulating organisms, which produce polycyclic aromatic hydrocarbons, and represent 22 - 26 \% of enriched sludge microbiota \cite{bengtsson2008production,lemaire2008microbial}. 

Little information beyond basic caracterisation of \emph{Stakelama pacifica} is availble, presumably because this worganisms was only proposed as it's own specied this year \cite{ogler2013description}.

\emph{Oleomonas} sp., identified with FISH and DGGE, constituted about 16 \% of the biomass in an upstream anaerobic bioreactor fed with brewery wastewater \cite{fernandez2008analysis}. Specifically \emph{Oleomonas sagaranensis} is involved in breakinf down urea \cite{kanamori2005allophanate,kanamori2004enzymatic}

\emph{Bacillus mucilaginosus} produces a bioflocculant which speeds up bioflocculation in starch wwt \cite{deng2003characteristics} and in a study designed to isolate strains with high lipolytic activity from olive mill wastewater, \emph{Bacillus} spp. exhibited the highest lipase activity \cite{ertuugrul2007isolation}.

\emph{Dechloromonas} sp. contituted 50 \% of beta proteo, most prevalent class,  in mature membr fouling biofilms of municipal wwtp \cite{miura2007membrane}. They degrade poly aromatic hydrocarbons \cite{oshiki2008pha}.

The rice pathogen \emph{Azospira oryzae} is also known by the synonym \emph{Dechlorosoma suillum} and uses perchlorate as a terminal electron acceptor  \cite{reinhold2000reassessment,tan2003dechlorosoma}. Also reduces toxic selenite and selenate to elemental selenium which can be sequestered from effluent before it is released into the environment \cite{hunter2007azospira,wilhelmus2013microbiological}.

Members of the genera \emph{Novosphingobium} have been shown to play an important role in wastewater remediation by degrading toxic dyes as well as estrogen - both which impact the ecosystem if released \cite{addison2007novosphingobium,hashimoto2009contribution}.

\emph{Roseomonas} sp. were found at about 5 \% abundance in activated sludge in a Chinese wastewater treatment plant and they can degrade organophosphate pesticides \cite{jiang2008bacterial,jiang2006isolation}.


 
\subsection{Lipid colonising isolates behave differently in the presence of lipids}
refer to diff in LC1 and background of isolates.. 


\subsection{depending on LC seq results}
While several papers have reported discovery of AHL mediated QS circuits outside the proteobacteria, closer inspection reveal that the assertions are not as definitive as originally suggested. Some reports of AHL QS in Archaea are based on weak ($\textgreater$ 35 \%) sequence identity of theoretical proteins to 'LuxI-like proteins', which is a descriptor for any enzyme from the acyltransferase superfamily where the majority of members are not involved in the QS circuit. This seems to be the case in the genome of \emph{Methanosalsum zhilinae} published in NCBI (Accession Number NC\_015676.1) and a study on \emph{Methanosaeta harundinacea} by Zhang et al. even ventured to call these putative genes \emph{luxI} orthologs \cite{zhang2012}. Sharif et al. report indication of an established and active AHL mediated QS system in the cyanobacterium \emph{Gloeothece sp.} on the basis of observing a change in EC protein production in response to AHL addition to the culture. They have not however established the molecular mechanism underlying these observations nor provided evidence for the presence of \emph{luxRI} homologs within the genome \cite{sharif2008}. 


Thorough investigation is required to firmly establish complete AHL QS circuits outside of the proteobacterial phylum. 

\end{document}