\documentclass[11pt]{article}
%\documentclass[draft]{article} for better debugging!

\usepackage{placeins}
\usepackage[square,sort,comma,numbers]{natbib}
\linespread{1.3}

\addtolength{\textwidth}{2cm}
\addtolength{\hoffset}{-1cm}


\addtolength{\textheight}{2cm}
\addtolength{\voffset}{-1cm}

\begin{document}
The function of our society depends on safe and efficient waste removal while sustainably recycling water and reducing environmental impact. The waste influent created by domestic households is extensive and fluctuates in composition. Hence the wastewater treatment system for influent remediation needs to be adapted to deal with the volume and variability. The lesser understood component employed by the treatment facilities is the biological aspect. This study explored the impact of lipids on activated sludge, the aerobic part of the biological treatment aspect. Ammonia is nitrified and organic matter, hence the oxygen demand of the mixed liquor, is reduced. Specifically it aimed to examine the effect of lipids on sludge morphology and change in microbial community structure as well as isolating lipid colonisers.

\subsection{Exposure to lipids changes the morphology of activated sludge}
It is evident that high lipid content in waste influent to activated sludge impacts the consistency of the sludge as well as the settleability. This impact has been widely reported (ref) but the concentration of lipid was found to be essential for determining the type of morphological change exhibited. During the trial enrichment of 1 \% GT, lipid droplets formed which were suspended throughout the sludge and decreased settleablity. When the GT concentration was increased to 2 \%, the sludge increased in viscosity instead of forming droplets and lightened in colour. This could be due to high concentration of fatty acids released, as the resulting sludge resembles an emulsion. The 2 \% lipid concentration for the enrichment was adapted from a lipase producing bacteria isolation study conducted by Haba et al \cite{haba2000isolation}, however the normal lipid concentration found in municipal wastewater is about 0.01 \% \cite{Forster_92}. Morphology that resembles droplets in lipid challenged activated sludge has been reported (ref), however neither the lipid concentration to elicit this response nor the underlying mechanisms are included. Secondary and tertiary transfer enrichment of the emulsion resulted in increased viscosity to an almost solid state and white appearance, which did not revert to anything resembling activated sludge at the last sampling point of 45 days of t3. No incidence of an emulsion like phenotype in the wastewater treatment plant setting has been found. Hence the lipid concentration that unbalances the biological treatment phase is between 0.01 and 1 \%. An accurate understanding of this value could pre-empt the deterioration of the biological treatment through efficient monitoring.

\subsection{Bacteria colonise lipids in activated sludge}
%Oleic acid is the most common fatty acid found in native as well as used olive and sunflower oils, which are commonly used in domestic settings \cite{haba2000isolation}. 
The direct colonisation of the lipid droplets was unexpected as this behaviour in AS sludge has not been recorded in the literature. However it has been recorded by Forster et al that the presence of lipids decreases flocculation and sludge settlement \cite{Forster_92}, the latter of which was observed in this experiment.

The microscopy pictures showcase the close association of biomass with the lipids. The aggregation around the droplets is akin to biofilms. Whether the dense mass of cells is surrounding and growing on the lipid or whether it is embedded within the lipid phase is unclear from this method. This should be further investigated by using scanning electron microscopy to gain an understanding of the nature of the call attachment to lipids.
Throughout the trial run a correlation between time and the decrease in lipid droplet size indicates that the microbes colonise the droplets and degrade their substratum as a C source. This observation complements the highest lipase activity recorded for membrane bound lipases, equally active in the control and enrichment replicates. The untreated samples recorded a higher lipase activity between days 9 to 22, which was not reflected in the total lipase activity for that time. This could be due to the difficulty of fraction separation that arose with the increased viscosity of the samples. On the 12th day sampling point the enriched samples were centrifuged for 1 hr at $0\,^{\circ}\mathrm{C}$, without successful separation of supernatant and flocs. 
Activated sludge contains a baseline lipid concentration, which could explain the maintenance of lipase for rapid C source degradation. Lipase activity in the EPS fraction almost doubled within the first 3 days, followed by steady decline while lipase activity in the supernatant fraction increases. Secreted lipase expressed by the \emph{$\gamma$}-proteobacterium \textit{Pseudomonas aeruginosa} associates with alginate within the EPS of flocs, which anchors the enzyme with weak bonding forces \cite{mayer1999role,wicker1987}. Hence the lipase  liberates fatty acids close to the cell. It follows that this mechanism is commonly employed and binding lipase to the EPS, and the delayed appearance of lipase in the supernatant fraction could be due to liberation from the hydrogen bonding and hence the floc association. The disassociation of lipase could indicate the dispersal phase in the floc life cycle.

If lipids seed flocs, the buoyancy of the lipid core could be responsible for lack of settlement rather than the presence of lipids preventing flocculation as suggested by Forster et el \citep{Forster_92}. Determining which microbial communities dominate during lipid induced floc settlement failure could provide another aspect for monitoring and anticipation of system failure.


As 2 \% GT causes the sludge to form an emulsion which is divergent from \emph{in situ} morphology and impacts the functionality of the lipase assay, 1 \% lipid should be used for future studies to create a setting with higher accuracy.
%who says lipidsprevent flocculation?

\subsection{Lipid colonising isolates behave differently in the presence of lipids}

Isolate LC5 was of interest due to it's high bioassay response. This suggests that this organism produces AHLs and that the production is different when the isolate is cultured in the presence or absence of lipid.
Mike - still workig through trying to find LuxI homologs for these.
%refer to diff in LC1 and background of isolates.. 

%While several papers have reported discovery of AHL mediated QS circuits outside the proteobacteria, closer inspection reveal that the assertions are not as definitive as originally suggested. Some reports of AHL QS in Archaea are based on weak ($\textgreater$ 35 \%) sequence identity of theoretical proteins to 'LuxI-like proteins', which is a descriptor for any enzyme from the acyltransferase superfamily where the majority of members are not involved in the QS circuit. This seems to be the case in the genome of \emph{Methanosalsum zhilinae} published in NCBI (Accession Number NC\_015676.1) and a study on \emph{Methanosaeta harundinacea} by Zhang et al. even ventured to call these putative genes \emph{luxI} orthologs \cite{zhang2012}. Sharif et al. report indication of an established and active AHL mediated QS system in the cyanobacterium \emph{Gloeothece sp.} on the basis of observing a change in EC protein production in response to AHL addition to the culture. They have not however established the molecular mechanism underlying these observations nor provided evidence for the presence of \emph{luxRI} homologs within the genome \cite{sharif2008}. Thorough investigation is required to firmly establish complete AHL QS circuits outside of the proteobacterial phylum. 

\subsection{The activated sludge microbiome changes in the presence of lipids}
DGGE is a technique that highlights change in microbial community over time and allows for excision of nucleic material for amplification \cite{yang2012evolution}. This is ideal for monitoring which microbes change in abundance over time in an enriched setting.



The initial time points for the enrichment culture replicates did not yield DNA extracts that could be utilised for further analysis from days 3 and 6 for the controls, as well as day 3 for the enriched culture. This is most likely due to the pH of the phenol solution - initially it was pH 4. This pH is suitable for RNA extraction while phenol at pH 8 is preferable for DNA extraction. The pH of the phenol solution was adjusted from day 9 onwards.

\subsubsection{\emph{Microbial community in replicate 4}}
It is evident from the DGGE community analysis of replicate 4 that lipids have a profound impact on the microbial community structure. The disparity between dominant organisms in the control vs. enriched cultures gives insight into the dominant players in lipid degradation. 
%band 1 bacteroidetes

The first half of the DGGE highlights the natively active microbiome in untreated activated sludge and the key organisms abundant in a functionally active wastewater remediating system. The following are all represented in high abundance throughout the 25 day sampling period. The bands in the control diminish over time as the intensity is directly proportional to abundance. The control was resupplied with filtered activated sludge supernatant so it follows that the community started starving after day 18 due do the lack of nutritional input and the abundance decreased.

%(inititally increased in intensity --> oligotrophs?)
Band 2 was identified as \emph{Bacillus} sp. which, in a study designed to isolate strains with high lipolytic activity from olive mill wastewater, exhibited the highest lipase activity \cite{ertuugrul2007isolation}. Specifically, \emph{Bacillus mucilaginosus} produces a bioflocculant which speeds up bioflocculation in starch wastewater treatment \cite{deng2003characteristics}. Whether this strain shares this expression pattern is unknown. However due to the it's recorded lipolytic activity, it is surprising that the abundance was not sustained in the enrichment.

Bands 3 and 6 are both designated as \emph{Dechloromonas} sp., but as two seperate species due to inter-species divergence of GC content of the region amplified. In a mature membrane fouling biofilms of municipal wastewater treatment in Japan, over 30 \% of the clones isolated were from the genus \emph{Dechloromonas} \cite{miura2007membrane}. This genera is also reported to degrade poly aromatic hydrocarbons \cite{oshiki2008pha}.

The rice pathogen \emph{Azospira oryzae} was appointed as the identity of band 4, otherwise known as \emph{Dechlorosoma suillum} \cite{tan2003dechlorosoma}. Members of this genera can use perchlorate as a terminal electron acceptor and reduces toxic selenite as well as selenate to elemental selenium which can be sequestered from effluent before it is released into the environment \cite{reinhold2000reassessment,hunter2007azospira,wilhelmus2013microbiological}.

%Band 5 xanthomonadales

The abundance of several genera increased with lipid enrichment and replaced the previously described genera active in the control.

Bands 7 and 8 Members of the genera \emph{Novosphingobium} have been shown to play an important role in wastewater remediation by degrading toxic dyes as well as estrogen - both which impact the ecosystem if released \cite{addison2007novosphingobium,hashimoto2009contribution}.


Bands 9, 10 and 11 \emph{Sphingomonas} sp. are present at about 5 - 10 \% in sludge as shown by FISH \cite{neef1999detection}, and play an important role in wastewater remediation. Members of this genera degrade testosterone and sterol hormones as well as the pollutant nonylphenol \cite{fujii2001sphingomonas,roh201017beta}.
% sphingomonas high GC varience paper


Band 12, \emph{Roseomonas} sp., was found at about 5 \% abundance in activated sludge in a Chinese wastewater treatment plant and they can degrade organophosphate pesticides \cite{jiang2008bacterial,jiang2006isolation}.

\subsubsection{\emph{Microbes in further enrichments}}
The further enrichment DGGE shows the communities dominant in several settings. It was noted that aggregates formed within the emulsions, they appeared solid suspended within the viscous cultures but disintegrated when probed and may be entirely colonised lipids. Hence the microbial community within these was of interest and one such aggregate was targeted for further secondary and tertiary enrichment. The aggregate expanded and developed tendrils upon further enrichment and    cells dispersed into the supernatant, which was tertiarily enriched, seperate to the aggregate. The tertiary enrichment time point (t3) is taken after 45 days to elucidate the microbiome of long term enrichment.


\emph{Nevskia} sp., band 1, is present in the aggregate at day 12 and then cease to exist at detectable levels at later time points for the aggregate. It is present in t3 aggregate supernatant, so it could have dissociated from the conglomerate to enter the supernatant phase. \emph{Nevskia} sp. are slow growing, where colony formation takes about a week, and they produce lipase \cite{kim2011nevskia}. A study by Chooklin et al endeavoured to isolate the most efficient surfactant producer from palm oil mill effluent for lipid remediation and found \emph{Nevskia} sp. \cite{chooklinutilization}.


Similarly, \emph{Sphingomonas} sp. from band 2 are present throughout day 12 to 18 but are absent in the t3 for the aggregate but do appear in the aggregate supernatant as band 10. This band was also assigned the identity of \emph{Sphingomonas} sp. by NCBI, albeit with a different accession number. The sequences used to identify band 2 and band 10 were aligned with BLAST, with lengths of 445 bp and 410 bp respectively. The sequence identity was 100 \% with an E-value of 0.0 hence the organisms that represent the two bands are regarded as synonymous regardless of differing accession numbers.
\emph{Sphingomonas} sp. also represent bands 4 and 8. The former is exclusive to primary aggregate sampling and disappears by the tertiary enrichment. The latter arises in t3 aggregate and emulsion.
Band 12, \emph{Sphingomonas} sp., appear in t3 emulsion. Whether the band in the 1 and 2 \% GT comparison is the same organisms is not possible to determine without sequencing as these bands are fractionally above the t3 bands.
From the DGGE gel, bands 2 and 13 appear to be the same organism. However during identity assignment, the latter returned as \emph{Stakelama pacifica} from NCBI with and E-value of 3e$^{-77}$ from a 163 bp sequence, while band 2 was concluded to be \emph{Sphingomonas} sp. with a more reliable E-value of 0.0 based on a 445 bp sequence. The two sequences were aligned with NCBI BLAST to investigate sequence similarity. The sequence alignment of band 13, which covers the 269 to 431 bp region of the amplified 16S rRNA fragment, aligns from the 269$^{th}$ bp of band 2 with a sequence identity was 100 \% with an E-value of 7e$^{-87}$. It is concluded that band 13 represents the \emph{Sphingomonas} from band 2 rather than \emph{Stakelama pacifica}.
% section about high GC content variability

\emph{Sphingobium} sp., the designation for band 3, is prevalent day 15 and 18 in the aggregate as well as day 7 in the secondary emulsion enrichment.


Band 5, \emph{Rhodovarius lipocyclicus}, is abundant from day 18 in the aggregate as well as in the tertiary enrichment and day 7 for t2 emulsion. However the highest abundance is recorded in the t3 aggregate supernatant, but it was not recorded in the tertiary emulsion enrichment. Limited information is available about \emph{Rhodovarius lipocyclicus}, beyond the basic information required for classification \cite{kampfer2004rhodovarius}.


\emph{Xanthobacter} sp., bands 6 and 7, are prevalent on day 18 in the aggregate. Band 7 also appears in the tertiary enrichment of the aggregate supernatant. As DGGE separates sequences by GC content, these may be two closely related species of the genus \emph{Xanthobacter} with slightly divergent GC content of the region amplified. This genus is reported to remediate aliphatic halogenated compounds which are commonly found in municipal wastewater \cite{janssen1985degradation} and have been shown to co-ordinate with \emph{Novosphingobium} sp. to degrade polyvinyl alcohol \cite{rong2009symbiotic}. \emph{Novosphingobium} sp. also degrade polycyclic aromatic hydrocarbons, which can by synthesised from lipid precursors \cite{addison2007novosphingobium}.


Members of the \emph{Bradyrhizobium} genera were designated as band 9 in the t3 aggregate supernatant and found in the emulsions t2 and t3. This genus is slow growing \cite{rebah2002wastewater} and usually associated with plant nodules. In plant symbiosis, they contribute by fixing nitrogen - an attribute required for successful A/A/O process. When exposed to activated sludge, these microbes have been shown to become highly antibiotic and heavy metal resistant.


Candidatus \emph{Competibacter} came up as band 11, only in the secondary emulsion enrichment and is a glycogen accumulating organisms, which produces polycyclic aromatic hydrocarbons, and represent 22 - 26 \% of enriched sludge microbiota \cite{bengtsson2008production,lemaire2008microbial}. 


Band 14 shows low abundance \emph{Oleomonas} sp., in t3 emulsion. However this genus was identified with FISH and DGGE, to constitute about 16 \% of the biomass in an upstream anaerobic bioreactor fed with brewery wastewater \cite{fernandez2008analysis}. Specifically \emph{Oleomonas sagaranensis} is involved in breaking down urea \cite{kanamori2005allophanate,kanamori2004enzymatic} which is essential for nitrification of ammonia in activated sludge.


While none of the bands from the 1 and 2 \% enrichment comparison were amplifiable for sequencing, the organisms active in these cultures were very similar. While the bands are present for both treatments, three of the bands show a difference in intensity and hence abundance.
mike - should I include letters for these bands on the DGGE pictre in results so I can refer to them?

\subsubsection{\emph{Common players}}
\emph{Sphingomonas} sp. are common in both replicate 4 enrichment as well as the various further enrichments.


While the majority of the microbes identified in the further enrichments, but not in the enrichment or control cultures for the 25 day duration of the enrichment experiment, these microorganisms are present in low abundance and become prevalent when their niche ability to degrade certain recalcitrant compounds becomes relevant. Also the detection of organisms was limited as several bands were not sequenceable.
% discuss ag plus ag SN combined = emulsion pattern
%point out the shitload of proteobacteria

\subsubsection{\emph{Comparison to core microbiomes}}
Wang et al found \emph{$\beta$-proteobacteria} to be most abundant in the 14 plants and bench top operations, closely followed by \emph{$\alpha$-proteobacteria}. While the most abundant class for this study was from the emph{$\alpha$-proteobacteria}, the order within the \emph{$\beta$-proteobacteria} class, Rhodocyclales, of which \emph{Dechloromonas} and \emph{Azospira} are part of, was the main emph{$\beta$-proteobacteria} as identifiedby Wang \cite{wang2012pyrosequencing}.


Wagner et al suggest fluorescence in situ hybridisation (FISH) with 16S rRNA  group-specific probes is necessary to accompany 16S rRNA sequencing sampling to ensure the accuracy of the OTU representation. However as this is a time consuming process, the majority of the community analysis is on 16S rRNA only \cite{Wagner_02} . A skewered representation of the OTUs in the community can arise from 16S primer bias which should be checked and accounted for.
% so check for bias!

Hesham et al compared the microbial communities of two differently operated municipal wastewater treatment plants over six months via DGGE. The 11 OTUs common to both plants, and the most abundant, were 18 \% to alpha-proteobacteria and 18 \% to beta-proteobacteria. The study demonstrates the similarities between microbial communities in the WWT plants and the adequacy of utilising DGGE for comparison \cite{Hesham_11}.
% yeah.. compare to the rest of the studies in table as well/
\bibliographystyle{acm}
\bibliography{thesis_ref}
\end{document}