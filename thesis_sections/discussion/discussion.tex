\documentclass[11pt]{article}
%\documentclass[draft]{article} for better debugging!

\usepackage{cite}

\linespread{1.3}

\addtolength{\textwidth}{2cm}
\addtolength{\hoffset}{-1cm}


\addtolength{\textheight}{2cm}
\addtolength{\voffset}{-1cm}

\begin{document}
While several papers have reported discovery of AHL mediated QS circuits outside the proteobacteria, closer inspection reveal that the assertions are not as definitive as originally suggested. Some reports of AHL QS in Archaea are based on weak ($\textgreater$ 35 \%) sequence identity of theoretical proteins to 'LuxI-like proteins', which is a descriptor for any enzyme from the acyltransferase superfamily where the majority of members are not involved in the QS circuit. This seems to be the case in the genome of \emph{Methanosalsum zhilinae} published in NCBI (Accession Number NC\_015676.1) and a study on \emph{Methanosaeta harundinacea} by Zhang et al. even ventured to call these putative genes \emph{luxI} orthologs \cite{zhang2012}. Sharif et al. report indication of an established and active AHL mediated QS system in the cyanobacterium \emph{Gloeothece sp.} on the basis of observing a change in EC protein production in response to AHL addition to the culture. They have not however established the molecular mechanism underlying these observations nor provided evidence for the presence of \emph{luxRI} homologs within the genome \cite{sharif2008}. 


Thorough investigation is required to firmly establish complete AHL QS circuits outside of the proteobacterial phylum. 
DGGE over pyrosequncing because it allows me to focus on the enriched bugs
yang 12 uses DGGE to track microbiome changes in ww!
wang 12 does pyroseq on diff wwtps in china. part of the dataset is initial COD! ompare to DGGE results. temp MOST IMPORTANT INFLUENCE shown by Wang 12, and within ref siggins and Wells.
\\However, as Wagner et al have suggested, fluorescence in situ hybridisation (FISH) with 16S rRNA  group-specific probes needs to accompany 16S rRNA sequencing sampling to ensure the accuracy of the OTU representation. However as this is a time consuming process, the majority of the community analysis is on 16S rRNA only \cite{Wagner_02} . A skewered representation of the OTUs in the community can arise from 16S primer bias which should be checked and accounted for.

The initial time points for the enrichment culture replicates did not yield DNA extracts that could be utilised for further analysis. This is most likely due to the pH of the phenols solution - initially it was pH 4. This pH is suitable for RNA extaction while phenol solution pH 8 is preferable for DNA extraction. This explains the high RNA but low DNA yield obtained upon DNA extraction.

Hesham et al compared the microbial communities of two differently operated municipal WWT plants over six months via DGGE. They 11 OTUs common to both plants, and the most abundant, were 18 \% to alpha-proteobacteria and 18 \% to beta-proteobacteria. The study demonstrates the similarities between microbial communities in the WWT plants and the adequacy of utilising DGGE for comparison \cite{Hesham_11}.
\end{document}